% ------------------------------------------------------------------
% RESUMO E ABSTRACT
%
% Para iniciarmos um ambiente, seja de resumo, abstract ou outro 
% qualquer, devemos começar pelo \begin. Onde eu escrevo nesse 
% arquivo, já me encontro dentro de um ``ambiente de resumo''.
%
% Aqui, apresentarei o resumo e mais um outro comando de exemplo: o 
% \lipsum. A principal função do \lipsum é gerar textos aleatórios -
% textos dummys. Assim, por mais que no presente documento - 
% no arquivo .TeX - os \lipsum[3-5] ou \lipsum[2-4] apareçam, no 
% .pdf esses só aparecerão como textos aleatórios.
% ------------------------------------------------------------------

%\setlength{\absparsep}{18pt} % ajusta o espaçamento dos parágrafos do resumo
\begin{resumo}
Este estudo apresenta uma comparação das expressões do Comitê de Política Monetária, levando em consideração as publicações das atas disponíveis após cada reunião, para o período de 2003 à 2018 -- isto é, do período de gestão de Meirelles ao período de gestão Goldfajn. A estrutura metodológica e analítica deste trabalho apresenta as técnicas de coleta e mineração de dados; a divergência de Kullback-Liebler; índices de sentimentos; e modelos de vetores auto-regressivos, bem como a função de resposta ao impulso. Os resultados deste trabalho indicam que efetivamente a forma de expressão da política monetária é relacionada com o período a ser analisado, sendo demonstrado que, de fato, a as distribuições de termos e palavras diferem de acordo com o cenário político brasileiro (PT-PMDB), bem como índices de sentimentos podem ser considerados em modelos macroeconômicos para predizer fenômenos econômicos.

\textbf{Palavras-chave}: Analise de sentimentos; Mineração de texto; COPOM; Kullback-Liebler.
\end{resumo}

% ------------------------------------------------------------------
% No caso do abstract, faremos a mesma coisa. Só adicionaremos a 
% opção abstract como o argumento do comando.
%-------------------------------------------------------------------

\begin{resumo}[ABSTRACT] % ESCREVER EM LETRAS MAIÚSCULAS
	
This study compares the expressions of the Monetary Policy Committee, taking into account the publications of the minutes available after each meeting, from 2003 to 2018 - that is, from the Meirelles management period to the Goldfajn management period. The methodological and analytical structure of this paper presents the techniques of data collection and mining; the Kullback-Liebler divergence; indices of feelings; and autoregressive vector models, as well as the impulse response function. The results of this work indicate that the form of monetary policy expression is effectively related to the period to be analyzed, showing that, in fact, the distributions of terms and words differ according to the Brazilian political scenario (PT-PMDB), as well as sentiment indices can be considered in macroeconomic models to predict economic phenomena.

\textbf{Keywords}: Sentiment analysis; Text mining; COPOM; Kullback-Liebler. 	
\end{resumo}




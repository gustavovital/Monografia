No artigo ``Measuring News Sentiment'' de \citeonline{shapiro2018measuring}, mais um índice de sentimentos foi proposto levando em consideração a estimação dos efeitos de positividade em artigos de forma mensal. Isto é, os autores consideram positividades em artigos em jornais de cunho econômico. 

Este índice de sentimento proposto foi utilizado como um exercício de aplicação, relacionando-o com a atividade econômica dos Estados Unidos. Neste exercício, é avaliado se especificamente a positividade do índice surte algum efeito na atividade econômica futura. Para isso foi utilizado o o método de projeção local proposto por \citeonline{jorda2005estimation}, similar a um vetor auto-regressivo (VAR), porém, menos restritivo. De forma geral, este método analisa como um choque do novo índice de sentimentos afeta um dado nível de atividade econômica. O novo choque do índice de sentimentos é construído como um componente da nova série de sentimentos que é ortogonal a atual e a seis defasagens de atividade econômica bem como a seis defasagens de si mesmo. Isso é, para cada previsão num horizonte $h$, uma regressão diferente é estimada para cada valor da atividade econômica calculada ($y_j$) no momento respectivo e defasado do novo índice de sentimentos e de outras quatro medidas econômicas (consumo, produção, taxa de juros real, e inflação) \cite{shapiro2018measuring}.

Feita a estimação, chega-se a conclusão que um choque positivo no índice de sentimentos, afeta positivamente o consumo, bem como na produção, e na taxa de juros real do FED. Houve, entretanto, uma leve redução para o nível de preços. O efeito no nível de preços é transitório, mas os efeitos no consumo, na produção e na taxa real dos fundos são mais duradouros, aumentando gradualmente até 12 meses após o choque, nas palavras de \citeonline{shapiro2018measuring}.


\begin{citacao}
``Estender o horizonte ainda mais [\dots] indica que as respostas de consumo, produção e taxa real atingem um pico entre 12 e 18 meses após o choque, antes de diminuir gradualmente.\footnote{Extending the horizon out further [\dots] indicates that the responses of consumption, output, and the real rate peak between 12 and 18 months after the shock before gradually waning.}'' \cite[p.19-20]{shapiro2018measuring}
\end{citacao}

Isso é, é possível avaliar uma notável melhora nas variáveis macroeconômicas, ainda, após os 12 meses a frente estimados no impulso resposta.

Em outro artigo \citeonline{shapiro2018measuring} tiveram como inspiração um segundo artigo de \citeonline{barsky2012information} -- que apresenta resultados semelhantes, foi verificado que um choque positivo de sentimentos leva a um aumento persistente em consumo, produção, e taxa real de juros; mas resulta em uma queda na inflação. A similaridades dos resultados, assim, fortalece a hipóteses que um possível índice de sentimentos tem medidas similares em impactos macroeconômicos, como por exemplo, o índice de sentimentos do consumidor.
